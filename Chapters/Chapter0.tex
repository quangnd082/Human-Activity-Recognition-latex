\chapter*{MỞ ĐẦU}
\addcontentsline{toc}{chapter}{MỞ ĐẦU}

\label{Chapter0} 

Computer vision đã và đang đóng góp một vai trò quan trọng trong tất cả các lĩnh vực trên toàn thế giới, mỗi năm có hàng ngàn báo cáo khoa học về đề tài này. Các công ty công nghệ hàng đầu như Google, Facebook, Amazon và Microsoft đang đầu tư mạnh mẽ vào học máy để phát triển các sản phẩm và dịch vụ mới. Ngoài ra, học máy cũng đang được sử dụng rộng rãi trong các lĩnh vực như y tế, tài chính, sản xuất và nhiều lĩnh vực khác để cải thiện hiệu suất và tối ưu hóa quy trình.

Những năm gần đây, khi mà khả năng tính toán của các máy tính được nâng lên một tầm cao mới và lượng dữ liệu khổng lồ được thu thập, Machine Learning đã tiến thêm một bước dài và Deep Learning ra đời. Deep Learning đã giúp máy tính thực thi những công việc phức tạp hơn như: phân loại cả ngàn vật thể trong các bức ảnh, tạo chú thích cho ảnh, bắt chước giọng nói và chữ viết của con người, giao tiếp với con người...

Nhận dạng hoạt động của con người (Human Activity Recognition – HAR) là một lĩnh vực nghiên cứu thú vị về thị giác máy tính và tương tác giữa người với máy.

HAR mang lại nhiều ứng dụng có ích cho cuộc sống con người. Theo dõi sức khỏe có thể được thực hiện thông qua các thiết bị đeo theo dõi hoạt động thể chất, nhịp tim và chất lượng giấc ngủ. Trong nhà thông minh, các giải pháp dựa trên HAR cho phép tiết kiệm năng lượng và tạo sự thoải mái cho cá nhân bằng cách phát hiện khi một người ra vào phòng và điều chỉnh ánh sáng hoặc nhiệt độ. Các thiết bị an toàn cá nhân có thể tự động cảnh báo các dịch vụ khẩn cấp hoặc một số liên lạc được chỉ định. Và đó chỉ là phần nổi của tảng băng chìm.

Với nhiều bộ dữ liệu có sẵn được chia sẻ công khai, việc tìm kiếm dữ liệu phục vụ mục đích nghiên cứu và phát triển HAR là rất đơn giản. Trong bài này này, Nhóm 6 sẽ cùng bạn tìm hiểu thêm về công nghệ tiên tiến nhất hiện nay của HAR, cùng với các phương pháp học sâu và bộ dữ liệu mở hỗ trợ tác vụ.


 

Nội dung chính của báo cáo có 5 chương:

Chương 1: Tổng quan về đề tài

Chương 2: Cơ sở lý thuyết dùng để giải quyết bài toán

Chương 3: Human Activity Recognition 

Chương 4: Demo sản phẩm và đánh giá các phương pháp

Chương 5: Kết luận  
 


